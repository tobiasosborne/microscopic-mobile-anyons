% lattice.tex
% LaTeX counterpart of docs/lattice.md
% Section §3.4

\section{Lattice Geometry}\label{sec:lattice}

\begin{assumption}\label{ass:lattice}
\begin{enumerate}[label=(A3.4.\arabic*)]
    \item Spatial dimension is 1.
    \item Open boundary conditions (no PBCs).
    \item Finite number of sites.
\end{enumerate}
\end{assumption}

\subsection{One-Dimensional Chain}\label{subsec:1d-chain}

\begin{definition}[Lattice]\label{def:lattice}
A \emph{lattice} $\Lambda$ is a finite set of sites. In this work:
\begin{equation}
    \Lambda = \{0, 1, 2, \ldots, n-1\}
\end{equation}
where $n = |\Lambda|$ is the number of sites.
\end{definition}

\begin{remark}
We use 0-based indexing for sites to align with physical position $x_j = \epsilon \cdot j$.
\end{remark}

\begin{definition}[Neighbour relation]\label{def:neighbour}
Sites $i$ and $j$ are \emph{neighbours} if $|i - j| = 1$. We write $i \sim j$.
\end{definition}

\begin{definition}[Boundary sites]\label{def:boundary}
The \emph{boundary} of $\Lambda$ consists of sites $0$ and $n-1$. The \emph{bulk} consists of sites $\{1, \ldots, n-2\}$.
\end{definition}

\subsection{Lattice Spacing and Physical Position}\label{subsec:spacing}

\begin{definition}[Physical length]\label{def:physical-length}
The \emph{physical length} of the system is $L > 0$.
\end{definition}

\begin{definition}[Lattice spacing]\label{def:lattice-spacing}
The \emph{lattice spacing} is
\begin{equation}
    \epsilon = \frac{L}{n}
\end{equation}
\end{definition}

\begin{definition}[Physical position]\label{def:physical-position}
The \emph{physical position} of site $j$ is
\begin{equation}
    x_j = \epsilon \cdot j = \frac{L \cdot j}{n}
\end{equation}
so that $x_0 = 0$ and $x_{n-1} = L(1 - 1/n) < L$.
\end{definition}

\begin{convention}
Where convenient, we set $\epsilon = 1$ (equivalently, $L = n$), so that site index equals physical position.
\end{convention}

\subsection{Open Boundary Conditions}\label{subsec:obc}

\begin{definition}[Open boundary conditions]\label{def:obc}
A system has \emph{open boundary conditions} (OBC) if:
\begin{enumerate}
    \item The lattice is a finite chain with distinct endpoints
    \item Site 0 has only one neighbour (site 1)
    \item Site $n-1$ has only one neighbour (site $n-2$)
\end{enumerate}
\end{definition}

\begin{remark}
This contrasts with \emph{periodic boundary conditions} (PBC) where site $n-1$ is also a neighbour of site 0. We exclude PBCs in this work (Assumption~A3.4.2).
\end{remark}

\begin{remark}
OBC implies:
\begin{itemize}
    \item No topological ground state degeneracy from nontrivial cycles
    \item Edge effects may be present
    \item Total charge is well-defined without ambiguity from winding
\end{itemize}
\end{remark}

\subsection{Local Hilbert Spaces}\label{subsec:local-hilbert}

\begin{definition}[Local Hilbert space]\label{def:local-hilbert}
Each site $j \in \Lambda$ carries a \emph{local Hilbert space} $\mathcal{H}_j$. For identical sites:
\begin{equation}
    \mathcal{H}_j \cong \mathcal{H}_{\mathrm{loc}} \quad \forall j
\end{equation}
\end{definition}

\begin{definition}[Local dimension]\label{def:local-dim}
The \emph{local dimension} is $d_{\mathrm{loc}} = \dim(\mathcal{H}_{\mathrm{loc}})$.
\end{definition}

\begin{definition}[Total Hilbert space]\label{def:total-hilbert-lattice}
The \emph{total Hilbert space} for the lattice is
\begin{equation}
    \mathcal{H}_{\mathrm{total}} = \bigotimes_{j=0}^{n-1} \mathcal{H}_j \cong \mathbb{C}^{d^n}
\end{equation}
\end{definition}

\begin{remark}
For mobile anyons, the local Hilbert space structure is more subtle---see \S\ref{sec:hilbert-space}. The ``local dimension'' depends on the occupation at that site.
\end{remark}

\subsection{Summary}

\begin{center}
\begin{tabular}{lll}
\toprule
\textbf{Concept} & \textbf{Symbol} & \textbf{Value/Definition} \\
\midrule
Number of sites & $n$ & $|\Lambda|$ \\
Site indices & $j$ & $0, 1, \ldots, n-1$ \\
Physical length & $L$ & System size \\
Lattice spacing & $\epsilon$ & $L/n$ \\
Physical position & $x_j$ & $\epsilon \cdot j$ \\
Local Hilbert space & $\mathcal{H}_j$ & Space at site $j$ \\
Local dimension & $d_{\mathrm{loc}}$ & $\dim(\mathcal{H}_{\mathrm{loc}})$ \\
Total dimension & --- & $d^n$ \\
\bottomrule
\end{tabular}
\end{center}

\subsection{Notation Conventions}

Throughout this project:
\begin{itemize}
    \item Sites are \textbf{0-indexed}: $j \in \{0, 1, \ldots, n-1\}$
    \item Boundary conditions are \textbf{open} (OBC)
    \item Default: $\epsilon = 1$ unless stated otherwise
    \item Tensor products are ordered left-to-right: $\mathcal{H}_0 \otimes \mathcal{H}_1 \otimes \cdots \otimes \mathcal{H}_{n-1}$
\end{itemize}
