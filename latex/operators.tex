% operators.tex
% LaTeX counterpart of docs/operators.md
% Section §4.3

\section{Operators as Morphisms}\label{sec:operators}

\subsection{Operator Space}\label{subsec:operator-space}

\begin{definition}[Operator]\label{def:operator}
An operator on $\mathcal{H}$ is a linear map $\mathcal{H} \to \mathcal{H}$.
\end{definition}

\begin{definition}[Operator as morphism sum]\label{def:operator-morphism}
Operators decompose as:
\begin{equation}
    \mathcal{A} \in \bigoplus_{A,B} \Mor(A, B)
\end{equation}
where $A, B$ range over tensor products of $\mathbf{1}, X_1, \ldots, X_{d-1}$.
\end{definition}

\subsection{Particle-Number Conservation}\label{subsec:particle-conservation}

\begin{definition}[Particle-conserving operator]\label{def:particle-conserving}
An operator $\mathcal{A}$ is \emph{particle-conserving} if it maps $\mathcal{H}_N \to \mathcal{H}_N$ for all $N$.
\end{definition}

\begin{definition}[Particle-changing operator]\label{def:particle-changing}
An operator with components in $\Mor(A, B)$ where $A$ and $B$ have different numbers of nontrivial factors changes particle number.
\end{definition}

\begin{example}
$\Mor(X_a \otimes X_b, X_c \otimes \mathbf{1})$ annihilates a particle (if $X_c$ is nontrivial) or two particles (if $X_c = \mathbf{1}$).
\end{example}

\subsection{Locality}\label{subsec:locality}

\begin{definition}[Local operator]\label{def:local-operator}
An operator is \emph{local} if it acts nontrivially only on a bounded number of adjacent sites.
\end{definition}

\begin{definition}[$k$-local operator]\label{def:k-local}
A \emph{$k$-local} operator acts on at most $k$ consecutive sites.
\end{definition}

For mobile anyons, a 2-local operator has components:
\begin{itemize}
    \item $\Mor(X_a \otimes X_b, X_c \otimes X_d)$ --- acts on two neighbouring anyons
    \item $\Mor(X_a \otimes \mathbf{1}, \mathbf{1} \otimes X_a)$ --- hops anyon right
    \item $\Mor(\mathbf{1} \otimes X_a, X_a \otimes \mathbf{1})$ --- hops anyon left
\end{itemize}
