% matrix_elements_2local.tex
% LaTeX counterpart of docs/matrix_elements_2local.md
% Section §4.5

\section{Matrix Elements of 2-Local Operators}\label{sec:matrix-elements}

\textbf{Planning ref:} \S4.5\\
\textbf{Status:} Draft

\begin{assumption}\label{ass:matrix-elements}
\begin{enumerate}[label=(A4.5.\arabic*)]
    \item Fusion category $\catC$ with unit $\one$ and simples $X_0=\one, X_1, \ldots, X_{d-1}$.
    \item Hard-core constraint: at most one anyon per site.
    \item Fusion tree basis fixed for $\catH_N^{(c)}$ (per Definition~\ref{def:fusion-tree}).
    \item Multiplicity-free fusion rules (i.e., $N_{ab}^c \in \{0, 1\}$).
\end{enumerate}
\end{assumption}

%=============================================================
\subsection{Fusion Trees as Basis Data}\label{subsec:fusion-tree-basis}
%=============================================================

\begin{definition}[Fusion tree basis states]\label{def:fusion-tree-basis-state}
A fusion tree $\tau$ (defined in Definition~\ref{def:fusion-tree}) specifies one of several basis choices for the morphism space $\Mor(X_c, \catO(\mathbf{j}, \mathbf{k}))$. We denote basis states as:
\begin{equation}
    |(\mathbf{j}, \mathbf{k}), \tau, c\rangle \in \catH_N^{(c)}
\end{equation}
where:
\begin{itemize}
    \item $(\mathbf{j}, \mathbf{k})$ is the configuration (site indices and anyon types)
    \item $\tau$ encodes the fusion tree structure
    \item $c$ is the total charge
\end{itemize}
\end{definition}

\begin{remark}
The fusion tree is not canonical; different choices of parenthesisation yield different bases. For multiplicity-free categories, all choices span the same space, but individual basis vectors depend on the choice.
\end{remark}

\begin{remark}
When two basis states have different fusion trees $\tau \neq \tau'$ (even for the same configuration and charge), they are orthogonal basis vectors. The Gram matrix between them is determined by the F-symbols of the category.
\end{remark}

%=============================================================
\subsection{Matrix Element Definitions}\label{subsec:matrix-element-def}
%=============================================================

\begin{definition}[Matrix element for 2-local morphism]\label{def:matrix-element-2local}
Given:
\begin{itemize}
    \item Basis states $|\psi\rangle = |(\mathbf{j}, \mathbf{k}), \tau, c\rangle$ and $|\phi\rangle = |(\mathbf{j}', \mathbf{k}'), \tau', c\rangle$ in $\catH_N^{(c)}$
    \item A morphism $f \in \Mor(X_a \otimes X_b, X_c \otimes X_d)$ acting on sites $m, m+1$
\end{itemize}
The matrix element is the complex number:
\begin{equation}
    \langle \phi | f_{m,m+1} | \psi \rangle \in \mathbb{C}
\end{equation}
Interpretation: $f_{m,m+1}$ acts as a morphism between the anyon types at sites $m, m+1$ in the source and target configurations.
\end{definition}

%=============================================================
\subsection{Type 1: Two-Particle Interaction}\label{subsec:two-particle}
%=============================================================

\begin{definition}[Two-particle interaction morphism]\label{def:two-particle-morph}
A morphism $f_{ab \to cd} \in \Mor(X_a \otimes X_b, X_c \otimes X_d)$ acts on two neighbouring sites with anyons of type $X_a, X_b$ and outputs $X_c, X_d$.
\end{definition}

\textbf{Configuration support:} Non-zero matrix elements occur only when:
\begin{itemize}
    \item Source configuration: $s_m = a, s_{m+1} = b$
    \item Target configuration: $s'_m = c, s'_{m+1} = d$
    \item All other sites identical: $s_i = s'_i$ for $i \notin \{m, m+1\}$
    \item Total charge preserved
\end{itemize}

In this case, the matrix element is:
\begin{equation}
    \langle (\mathbf{j}', \mathbf{k}'), \tau' | f_{ab \to cd} | (\mathbf{j}, \mathbf{k}), \tau \rangle = \alpha_{ab \to cd}(\tau, \tau')
\end{equation}
where $\alpha_{ab \to cd}(\tau, \tau')$ is determined by the morphism and the fusion tree compatibility.

\begin{remark}[Multiplicity-free case]
$N_{ab}^c \in \{0,1\}$ for all triples, so the morphism space $\Mor(X_a \otimes X_b, X_c \otimes X_d)$ is either $\{0\}$ or one-dimensional. If non-zero, the coefficient $\alpha_{ab \to cd}$ is a phase/scalar set by normalisation.
\end{remark}

\begin{remark}[Forbidden fusion]
If $N_{ab}^c = 0$ (fusion forbidden), then $\Mor(X_a \otimes X_b, X_c) = \{0\}$, and the operator component is identically zero.
\end{remark}

\begin{claim}[Hermiticity]\label{claim:two-particle-hermitian}
If $f_{ab \to cd}$ is a unitary morphism (or paired with its adjoint), the physical operator $\catO = f_{m,m+1} + f_{m,m+1}^\dagger$ is Hermitian and preserves the sector $\catH_N^{(c)}$.
\end{claim}

%=============================================================
\subsection{Type 2: Right Hopping}\label{subsec:hop-right}
%=============================================================

\begin{definition}[Right-hopping morphism]\label{def:hop-right}
A morphism $h_R \in \Mor(X_a \otimes \one, \one \otimes X_a)$ describes translating an anyon of type $X_a$ from site $m$ to site $m+1$.
\end{definition}

\textbf{Hard-core configuration support:}
\begin{itemize}
    \item Source: anyon $X_a$ at site $m$, vacuum ($\one$) at site $m+1$
    \item Target: vacuum at site $m$, anyon $X_a$ at site $m+1$
\end{itemize}

Matrix element:
\begin{equation}
    \langle (\mathbf{j}', \mathbf{k}'), \tau' | h_{R,m,m+1} | (\mathbf{j}, \mathbf{k}), \tau \rangle = \begin{cases}
    \beta_a(\tau, \tau') & \text{if } s_m = a, s_{m+1} = 0 \text{ and } s'_m = 0, s'_{m+1} = a \\
    0 & \text{otherwise}
    \end{cases}
\end{equation}
where $\beta_a(\tau, \tau')$ is determined by the fusion tree evolution and normalisation convention.

\begin{claim}[Locality of hopping]\label{claim:hop-locality}
Right hopping is a 2-local operator: it couples only configurations differing by a single anyon displacement at neighbouring sites.
\end{claim}

%=============================================================
\subsection{Type 3: Left Hopping}\label{subsec:hop-left}
%=============================================================

\begin{definition}[Left-hopping morphism]\label{def:hop-left}
A morphism $h_L \in \Mor(\one \otimes X_a, X_a \otimes \one)$ describes translating an anyon of type $X_a$ from site $m+1$ to site $m$.
\end{definition}

\textbf{Hard-core configuration support:}
\begin{itemize}
    \item Source: vacuum at site $m$, anyon $X_a$ at site $m+1$
    \item Target: anyon $X_a$ at site $m$, vacuum at site $m+1$
\end{itemize}

\begin{remark}[Relation to right hopping]
By rigidity of the fusion category, $h_L$ and $h_R$ are related via duality. In particular, if both use consistent normalisation for $X_a \otimes \one \leftrightarrow \one \otimes X_a$, then $h_L = h_R^\dagger$ (up to phase).
\end{remark}

%=============================================================
\subsection{Operator Matrix Representation}\label{subsec:operator-matrix}
%=============================================================

\begin{definition}[Full operator matrix in $\catH_N^{(c)}$]\label{def:operator-matrix}
Let $\{|\psi_i\rangle = |(\mathbf{j}_i, \mathbf{k}_i), \tau_i, c\rangle\}_{i=1}^{d_N^{(c)}}$ be an orthonormal basis of $\catH_N^{(c)}$ (in a chosen fusion tree basis). The matrix representation of a 2-local operator $\catO$ is:
\begin{equation}
    [\catO]_{ij} := \langle \psi_i | \catO | \psi_j \rangle \in \mathbb{C}
\end{equation}
\end{definition}

\begin{remark}
This is basis-dependent; different fusion tree bases yield different matrix coordinates. However, the eigenvalues and trace (and all basis-independent properties) are invariant.
\end{remark}

\begin{claim}[Sparsity of 2-local operators]\label{claim:sparsity}
For 2-local operators in the hard-core sector:
\begin{itemize}
    \item Type 1 (two-particle interaction) couples configurations with anyons at the same pair of sites $(m, m+1)$; expect $O(n)$ nonzero entries per row.
    \item Type 2, 3 (hopping) couple configurations differing by a single displacement; expect $O(1)$ nonzero entries per row.
\end{itemize}
Total matrix sparsity: $O(n \cdot d_N^{(c)})$ nonzero entries.
\end{claim}

%=============================================================
\subsection{Example: Two Fibonacci Anyons}\label{subsec:fibonacci-example}
%=============================================================

\begin{example}[Fibonacci $\tau$ anyons, $N=2$ on 3 sites]\label{ex:fibonacci-2-anyons}
Fusion category: Fibonacci, $d=2$ (objects $\one, \tau$), $\tau \otimes \tau = \one \oplus \tau$ (multiplicity 1 each).

Hard-core configurations with 2 anyons on 3 sites: $\mathrm{Conf}_2^{\mathrm{HC}}(3) = \{(0,1), (0,2), (1,2)\}$ (pairs of occupied sites).

Basis states in $\catH_2^{(\one)}$ (two $\tau$ anyons fusing to $\one$):
\begin{itemize}
    \item $|\psi_1\rangle = |(0,1), \tau, \one\rangle$ = anyons at sites 0, 1 fusing to $\one$
    \item $|\psi_2\rangle = |(0,2), \tau, \one\rangle$ = anyons at sites 0, 2 fusing to $\one$
    \item $|\psi_3\rangle = |(1,2), \tau, \one\rangle$ = anyons at sites 1, 2 fusing to $\one$
\end{itemize}

Right-hopping operator $h_{R,0,1}$ on sites 0, 1 (moves anyon from site 0 to site 1):
\begin{itemize}
    \item Source: anyon at site 0, vacuum at site 1 $\to$ only $|\psi_2\rangle = |(0,2)\rangle$ qualifies
    \item Target: vacuum at site 0, anyon at site 1 $\to$ only $|\psi_3\rangle = |(1,2)\rangle$ qualifies
    \item Non-zero element: $\langle \psi_3 | h_{R,0,1} | \psi_2 \rangle$
\end{itemize}

Matrix of $h_{R,0,1}$ in basis $\{\psi_1, \psi_2, \psi_3\}$:
\begin{equation}
    h_{R,0,1} = \begin{pmatrix} 0 & \beta & 0 \\ 0 & 0 & 0 \\ 0 & 0 & 0 \end{pmatrix}
\end{equation}
where $\beta$ is the amplitude (normalisation-dependent).
\end{example}

%=============================================================
\subsection{Critical Observations and Open Questions}\label{subsec:observations}
%=============================================================

\textbf{Observation 1} (Fusion tree dependence). The matrix elements explicitly depend on the choice of fusion tree basis via $\tau, \tau'$. This choice is not canonical for multiplicity $N_{ab}^c > 1$. All published definitions must remain basis-independent; fusion trees are a computational tool only.

\textbf{Observation 2} (Missing fusion tree reduction). Computing actual matrix elements requires:
\begin{enumerate}
    \item Extracting the local fusion tree structure at sites $j, j+1$ from $\tau$
    \item Composing with the morphism $f$ or $h$
    \item Tracking how $\tau$ evolves to $\tau'$
    \item Applying R-matrices if the operator causes anyon braiding
\end{enumerate}
This requires implementation of fusion tree reduction coefficients and F-symbols.

\textbf{Question 4.5.1} (Hard-core + arbitrary charge). For $N$ anyons with definite charge $c$ in hard-core sector, is the dimension of $\catH_N^{(c)}$ always $O\left(\binom{n}{N}\right)$? What are the constraints on $c$?

\textbf{Question 4.5.2} (Multiplicity handling). How do matrix elements change if $N_{ab}^c > 1$? Each morphism space becomes multi-dimensional, introducing index-dependent amplitudes $\alpha^{(\mu)}_{ab \to cd}$.

\textbf{Question 4.5.3} (Normalisation convention). What is the standard normalisation for $h_R$ and $h_L$? Should they be unitary? Hermitian? How does this interact with R-matrices and the fusion category structure?
