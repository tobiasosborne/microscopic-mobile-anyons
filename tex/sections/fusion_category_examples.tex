% fusion_category_examples.tex
% LaTeX counterpart of docs/fusion_category_examples.md
% Section §3.1.8

\section{Examples of Fusion Categories}\label{sec:fusion-examples}

This section enumerates concrete fusion categories used in mobile anyon models. Each example specifies fusion rules, F-symbols, R-symbols (if braided), and the simple object count (rank).

\begin{assumption}\label{ass:fusion-examples}
\begin{enumerate}[label=(A3.1.8.\arabic*)]
    \item Fusion categories listed here are semisimple and rigid.
    \item Numerical F/R-symbols are computed via standard references (Kitaev, Rowell).
\end{enumerate}
\end{assumption}

%=============================================================
\subsection{Fibonacci Category (Multiplicity-Free)}\label{subsec:fibonacci}
%=============================================================

\textbf{Category:} $\mathcal{C}_{\mathrm{Fib}}$ \\
\textbf{Rank:} $d = 2$ \\
\textbf{Simple objects:} $\one = X_0$, $\tau = X_1$

\subsubsection{Fusion Rules}

\begin{center}
\begin{tabular}{c|cc}
$\otimes$ & $\one$ & $\tau$ \\
\hline
$\one$ & $\one$ & $\tau$ \\
$\tau$ & $\tau$ & $\one \oplus \tau$
\end{tabular}
\end{center}

\textbf{Multiplicities:} $N_{\tau\tau}^{\one} = N_{\tau\tau}^{\tau} = 1$ (multiplicity-free).

\subsubsection{Quantum Dimensions}

\begin{itemize}
    \item $d_{\one} = 1$
    \item $d_\tau = \phi = \frac{1+\sqrt{5}}{2}$ (golden ratio)
\end{itemize}

\subsubsection{F-Symbols (Associator)}

For the only nontrivial fusion channel $\tau \otimes \tau \to \one \oplus \tau$:
\begin{equation}
    F_{\tau,\tau,\tau}^\tau = \begin{pmatrix} \phi^{-1} & \phi^{-1/2} \\ \phi^{-1/2} & -\phi^{-1} \end{pmatrix}
\end{equation}
where rows/columns are indexed by the intermediate fusion channel $e \in \{\one, \tau\}$, and the matrix connects the two fusion orders: $(\tau \otimes \tau) \otimes \tau$ vs.\ $\tau \otimes (\tau \otimes \tau)$. This matrix is unitary: $F^\dagger F = I$.

\textbf{Numerical values:} $\phi^{-1} = \phi - 1 \approx 0.6180$, $\phi^{-1/2} \approx 0.7861$ \cite{Kitaev2006} \verified

\subsubsection{R-Symbols (Braiding)}

For the \emph{braided} Fibonacci category (adding R-symbols to the fusion category):
\begin{equation}
    R_{\tau,\tau}^{\one} = e^{4\pi i/5}, \qquad R_{\tau,\tau}^{\tau} = e^{-3\pi i/5}
\end{equation}
The topological spin (twist) of $\tau$ is $\theta_\tau = e^{4\pi i/5}$, corresponding to conformal weight $h_\tau = 2/5$. \verified

%=============================================================
\subsection{Ising Category (Multiplicity-Free)}\label{subsec:ising}
%=============================================================

\textbf{Category:} $\mathcal{C}_{\mathrm{Ising}}$ \\
\textbf{Rank:} $d = 3$ \\
\textbf{Simple objects:} $\one = X_0$, $\sigma = X_1$, $\psi = X_2$

\subsubsection{Fusion Rules}

\begin{center}
\begin{tabular}{c|ccc}
$\otimes$ & $\one$ & $\sigma$ & $\psi$ \\
\hline
$\one$ & $\one$ & $\sigma$ & $\psi$ \\
$\sigma$ & $\sigma$ & $\one \oplus \psi$ & $\sigma$ \\
$\psi$ & $\psi$ & $\sigma$ & $\one$
\end{tabular}
\end{center}

\textbf{Multiplicities:} All fusion coefficients are $0$ or $1$ (multiplicity-free).

\subsubsection{Quantum Dimensions}

\begin{itemize}
    \item $d_{\one} = 1$
    \item $d_\sigma = \sqrt{2}$
    \item $d_\psi = 1$
\end{itemize}

\subsubsection{F-Symbols}

Non-trivial associators exist for $\sigma \otimes \sigma \to \one \oplus \psi$:
\begin{equation}
    F_{\sigma,\sigma,\sigma}^\sigma = \frac{1}{\sqrt{2}} \begin{pmatrix} 1 & 1 \\ 1 & -1 \end{pmatrix}
\end{equation}
This is the Hadamard matrix, which is unitary. \verified

\subsubsection{R-Symbols}

For braided (modular) Ising category:
\begin{equation}
    R_{\sigma,\sigma}^{\one} = e^{i\pi/8}, \quad R_{\sigma,\sigma}^{\psi} = e^{-3i\pi/8}
\end{equation}
The topological spin of $\sigma$ is $\theta_\sigma = e^{i\pi/8}$. \verified

%=============================================================
\subsection{$\mathbb{Z}_N$ Categories (Pointed)}\label{subsec:zn}
%=============================================================

\textbf{Category:} $\mathcal{C}_{\mathbb{Z}_N}$ \\
\textbf{Rank:} $d = N$ \\
\textbf{Simple objects:} $X_0, X_1, \ldots, X_{N-1}$ (cyclic group)

\subsubsection{Fusion Rules}

\begin{equation}
    X_a \otimes X_b = X_{(a+b) \bmod N}
\end{equation}

\textbf{Multiplicities:} All $N_{ab}^c = 0$ or $1$ (multiplicity-free for standard abelian fusion).

\subsubsection{Quantum Dimensions}

$d_{X_a} = 1$ for all $a$. (For abelian/pointed categories, all objects are invertible and hence have quantum dimension $1$; see EGNO \S8.4 \cite{EGNO2015}.)

\subsubsection{F-Symbols}

For abelian (group-like) fusion, all nontrivial associators are trivial:
\begin{equation}
    F_{a,b,c}^e = 1
\end{equation}

\subsubsection{R-Symbols}

Braiding given by a 2-cocycle $\sigma(a, b) \in U(1)$:
\begin{equation}
    R_{a,b}^{a+b} = \sigma(a, b)
\end{equation}

\begin{example}[$\mathbb{Z}_2$ with fermionic statistics]
\begin{equation}
    R_{\mathrm{f}, \mathrm{f}}^{\one} = -1
\end{equation}
\end{example}

%=============================================================
\subsection{sVec Category (Fermionic)}\label{subsec:svec}
%=============================================================

\textbf{Category:} $\mathcal{C}_{\mathrm{sVec}}$ \\
\textbf{Rank:} $d = 2$ \\
\textbf{Simple objects:} $\one = X_0$ (boson), $\psi = X_1$ (fermion)

\subsubsection{Fusion Rules}

\begin{center}
\begin{tabular}{c|cc}
$\otimes$ & $\one$ & $\psi$ \\
\hline
$\one$ & $\one$ & $\psi$ \\
$\psi$ & $\psi$ & $\one$
\end{tabular}
\end{center}

\textbf{Multiplicities:} Multiplicity-free ($N_{\psi\psi}^{\one} = 1$).

\subsubsection{Quantum Dimensions}

\begin{itemize}
    \item $d_{\one} = 1$
    \item $d_\psi = 1$ (fermionic dimension)
\end{itemize}

\subsubsection{F-Symbols}

For fermionic fusion (super-case), the associator has an extra sign:
\begin{equation}
    F_{\psi,\psi,\psi}^{\one} = 1 \quad \text{(standard)}
\end{equation}
But crossing rules differ due to Fermi statistics.

\subsubsection{R-Symbols}

\begin{equation}
    R_{\psi,\psi}^{\one} = e^{i\pi} = -1
\end{equation}
(fermionic exchange is anticommuting; eigenvalue is $-1$).

%=============================================================
\subsection{Categories with Multiplicities}\label{subsec:multiplicity-examples}
%=============================================================

Most physically relevant fusion categories are multiplicity-free ($N_{ab}^c \in \{0,1\}$), but multiplicities arise in important examples.

\subsubsection{$SU(2)_k$ for $k \geq 3$}

\textbf{Category:} $\mathcal{C}_{SU(2)_k}$ (level-$k$ truncation of $SU(2)$ representations) \\
\textbf{Rank:} $d = k+1$ \\
\textbf{Simple objects:} $V_0, V_1, \ldots, V_k$ (spin-$j/2$ representations with $j = 0, 1, \ldots, k$)

\textbf{Fusion rules:} Truncated Clebsch--Gordan:
\begin{equation}
    V_a \otimes V_b = \bigoplus_{c = |a-b|}^{\min(a+b, 2k-a-b)} V_c
\end{equation}
where the sum runs in steps of $2$.

\textbf{Multiplicities:} For $k \geq 3$, we have $N_{ab}^c > 1$ in some cases. For instance, in $SU(2)_4$:
\begin{equation}
    V_2 \otimes V_2 = V_0 \oplus V_2 \oplus V_4
\end{equation}
is multiplicity-free, but higher levels exhibit multiplicities.

\begin{remark}[Other multiplicity examples]
Categories with multiplicities include:
\begin{itemize}
    \item Haagerup categories $\mathcal{H}_1, \mathcal{H}_2, \mathcal{H}_3$ (exotic fusion categories)
    \item Tambara--Yamagami categories for non-abelian groups
    \item Representation categories of quantum groups at roots of unity
\end{itemize}
The general framework developed in this paper handles multiplicities via multiplicity indices (Remark~\ref{rem:multiplicity-indices}).
\end{remark}

%=============================================================
\subsection{Summary Table}\label{subsec:examples-summary}
%=============================================================

\begin{table}[h]
\centering
\begin{tabular}{lccccl}
\toprule
\textbf{Category} & \textbf{Rank} & \textbf{Mult-Free?} & \textbf{Braided?} & \textbf{Modular?} & \textbf{Status} \\
\midrule
Fibonacci & 2 & $\checkmark$ & $\checkmark$ & $\checkmark$ & Implemented \\
Ising & 3 & $\checkmark$ & $\checkmark$ & $\checkmark$ & Implemented \\
$\mathbb{Z}_N$ & $N$ & $\checkmark$ & $\checkmark$ & $\times$ & Template \\
sVec & 2 & $\checkmark$ & $\checkmark$ & $\times$ & Fermionic \\
$SU(2)_k$ & $k+1$ & $\times$ (for $k \geq 3$) & $\checkmark$ & $\checkmark$ & Reference \\
\bottomrule
\end{tabular}
\caption{Summary of fusion category examples.}
\label{tab:examples-summary}
\end{table}

\begin{remark}
\textbf{Unverified:} All F/R-symbol values marked \unverified\ pending HITL review against literature.
\end{remark}

\begin{remark}
\textbf{Not exhaustive:} Other fusion categories (e.g., Haagerup, $SU(2)_k$, Tambara--Yamagami) can be added as needed. Note: Potts models and Virasoro algebras are related CFT structures but are not themselves fusion categories.
\end{remark}
