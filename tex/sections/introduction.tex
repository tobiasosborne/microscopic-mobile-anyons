% introduction.tex
% Section 2: Introduction

\section{Introduction}
\label{sec:introduction}

\subsection{Why Mobile Anyons?}
\label{sec:why-mobile-anyons}

Anyonic particles---quasiparticle excitations obeying exotic exchange statistics that interpolate between bosonic and fermionic behaviour---have become central objects of study in condensed matter physics and quantum information theory. The mathematical framework of fusion categories provides a rigorous foundation for describing systems of anyons, encoding their fusion rules, braiding statistics, and algebraic structure in a unified formalism~\cite{Kitaev2006,ENO2005}.

Existing microscopic many-body models based on fusion categories, such as the celebrated golden chain~\cite{Feiguin2007}, describe systems where anyons occupy fixed positions on a lattice. In such models, the number of anyons is constant and their positions are predetermined---analogous to the tightly-packed Mott insulating phase in condensed matter systems. While these models have yielded profound insights into anyonic physics, including connections to conformal field theory at criticality, they represent only a special limiting case of the broader landscape of possible anyonic phases.

In realistic physical systems, however, particle number and position are typically dynamical degrees of freedom. Electrons in metals move freely through the lattice; ultracold atoms in optical lattices tunnel between sites; quasiparticles in quantum Hall systems can be created, annihilated, and transported. A complete understanding of anyonic matter requires extending the categorical framework to accommodate such \emph{mobile} or \emph{itinerant} anyons.

The motivation for developing microscopic models of mobile anyons is threefold:

\begin{enumerate}[label=(\roman*)]
    \item \textbf{Connection to realistic physics.} Physical systems supporting anyonic excitations---such as fractional quantum Hall states or topological superconductors---generically allow for variable anyon number and mobility. Microscopic models capturing these features would bridge the gap between abstract categorical data and experimentally accessible phenomena including transport, scattering, and thermodynamic properties.

    \item \textbf{Exploration of new phases.} Beyond the Mott-like limit of fixed anyons, one expects a rich phase diagram including dilute anyonic gases, anyonic superfluids, and intermediate correlated phases. The interplay between anyonic exchange statistics and spatial dynamics may give rise to novel collective phenomena without analogue in conventional bosonic or fermionic systems.

    \item \textbf{Recovery of known limits.} A consistent framework must reproduce well-understood special cases. When fusion rules are trivial ($X \otimes X = \one$), we should recover ordinary bosons or fermions depending on the braiding structure. For super-vector spaces (sVec), the construction should reduce exactly to fermionic Fock space with standard anticommutation relations. For dense configurations, we should recover the physics of the golden chain and related models.
\end{enumerate}

The present work addresses this gap by developing a systematic framework for constructing microscopic lattice models describing mobile anyons arising from an arbitrary fusion category. Working in a first-quantised formalism on a one-dimensional chain with open boundary conditions, we construct Hilbert spaces that accommodate variable anyon number and anyon mobility, and define microscopic Hamiltonians for physically motivated scenarios.

\subsection{Literature Gap}
\label{sec:literature-gap}

The theoretical study of anyonic systems has developed along two largely disjoint lines. On the mathematical side, the theory of fusion categories~\cite{ENO2005,EGNO2015} provides a complete algebraic framework for describing anyon types, their fusion rules, and braiding statistics. This framework is abstract and basis-independent, applicable to any system whose excitations form a fusion category. On the physical side, microscopic lattice models---most prominently the golden chain~\cite{Feiguin2007} and its generalisations---have demonstrated how to construct explicit Hamiltonians whose low-energy physics is governed by anyonic degrees of freedom.

However, these microscopic models invariably assume a \emph{dense} configuration: one anyon occupies each lattice site, and the Hilbert space is spanned by the different ways adjacent anyons can fuse. This corresponds to a Mott-insulating regime where particle number is maximal and fixed, and spatial dynamics are frozen out. The only degrees of freedom are the internal fusion channels---essentially, which superselection sector the system occupies locally.

Several important physical questions lie outside this dense limit:
\begin{itemize}
    \item \textbf{Dilute anyonic gases.} What is the ground state of a system with fewer anyons than sites? How do anyonic statistics affect spatial correlations in a dilute gas?
    \item \textbf{Hopping and transport.} If anyons can tunnel between sites, how does the interplay of mobility and exotic statistics manifest in transport properties?
    \item \textbf{Variable particle number.} Can we define pair-creation and annihilation processes consistent with fusion rules? What phases emerge when particle number fluctuates?
    \item \textbf{Interpolation to known limits.} A framework for mobile anyons should reduce to standard fermionic or bosonic physics when the fusion category is trivial (sVec or Vec respectively), providing a consistency check and physical intuition.
\end{itemize}

To our knowledge, no systematic framework exists for constructing microscopic lattice models of mobile anyons arising from a general fusion category. The present work aims to fill this gap.

% Placeholder for subsequent subsections
% \subsection{Contributions of This Work}
% \label{sec:contributions}
%
% \subsection{Paper Outline}
% \label{sec:paper-outline}
