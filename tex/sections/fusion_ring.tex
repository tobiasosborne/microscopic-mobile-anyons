% fusion_ring.tex
% LaTeX counterpart of docs/fusion_ring.md
% Section §3.1.1

\section{Fusion Ring}\label{sec:fusion-ring}

\begin{assumption}\label{ass:fusion-ring}
\begin{enumerate}[label=(A\arabic*)]
    \item Finite set of simple objects $\{X_i\}_{i=0}^{d_{\mathcal{C}}-1}$.
    \item Structure constants $N_{ab}^c \in \mathbb{Z}_{\ge 0}$ are associative and unital with unit $\mathbf{1}$.
\end{enumerate}
\end{assumption}

\subsection{Simple Objects}\label{subsec:simple-objects}

\begin{definition}[Simple object]\label{def:simple-object}
Let $\mathcal{C}$ be a fusion category over field $\mathbb{C}$. An object $X \in \mathcal{C}$ is \emph{simple} if it satisfies all three conditions:
\begin{enumerate}
    \item \textbf{Nonzero:} $X \neq 0$ (in the sense that $X$ is not the zero object).
    \item \textbf{Indecomposable:} If $X \cong Y \oplus Z$, then $Y = 0$ or $Z = 0$.
    \item \textbf{Schur:} $\End(X) \cong \mathbb{C}$ (all endomorphisms are scalar multiples of the identity).
\end{enumerate}
\end{definition}

\begin{consequence}\label{cons:semisimple-decomp}
By semisimplicity of fusion categories (Deligne's theorem), every object $A \in \mathcal{C}$ decomposes as a finite direct sum of simple objects:
\begin{equation}
    A \cong \bigoplus_{i \in I} X_i^{\oplus m_i}
\end{equation}
where $X_i$ are simple and $m_i \in \mathbb{Z}_{\ge 0}$ are multiplicities.
\end{consequence}

\begin{remark}
For our purposes, we work with fusion categories where the simple objects are \emph{distinguishable} by their labels: $\{X_0, X_1, \ldots, X_{d_{\mathcal{C}}-1}\}$ with $X_0 = \mathbf{1}$ (the tensor unit/vacuum).
\end{remark}

\begin{citationblock}
Etingof--Nikshych--Ostrik, \emph{Ann.\ Math.}\ \textbf{162} (2005), Theorem~2.7 \cite{ENO2005} \unverified
\end{citationblock}

\subsection{Fusion Ring Definition}\label{subsec:fusion-ring-def}

\begin{definition}[Fusion ring]\label{def:fusion-ring}
A \emph{fusion ring} is a finitely generated free abelian group $R = \bigoplus_{i \in I} \mathbb{Z} X_i$ with a ring structure satisfying:
\begin{enumerate}
    \item $X_0 = \mathbf{1}$ is the unit element.
    \item The product of basis elements satisfies
    \begin{equation}\label{eq:fusion-product}
        X_i X_j = \sum_{k\in I} N_{ij}^k X_k,
    \end{equation}
    where $N_{ij}^k \in \mathbb{Z}_{\ge 0}$ are the \emph{fusion coefficients} (or fusion multiplicities).
    \item There exists an involution $i \mapsto i^*$ such that
    \begin{equation}\label{eq:duality-condition}
        N_{ij}^0 = \delta_{i, j^*}.
    \end{equation}
\end{enumerate}
The involution gives duality: $X_i^* = X_{i^*}$. Associativity follows from the ring axioms:
\begin{equation}\label{eq:fusion-associativity}
    \sum_e N_{ij}^e N_{ek}^\ell = \sum_e N_{jk}^e N_{ie}^\ell \quad \text{for all } i, j, k, \ell \in I.
\end{equation}
\end{definition}

\begin{remark}
Fusion rings are generally \emph{not commutative}, i.e., $N_{ij}^k \neq N_{ji}^k$ in general.
\end{remark}

\begin{citationblock}
Etingof--Nikshych--Ostrik, \emph{Ann.\ Math.}\ \textbf{162} (2005), 581--642, Def.~3.1 \cite{ENO2005} \unverified
\end{citationblock}
