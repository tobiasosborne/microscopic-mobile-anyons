% hardcore_softcore.tex
% LaTeX counterpart of docs/hardcore_softcore.md
% Section §4.4

\section{Hard-Core vs Soft-Core Anyons}\label{sec:hardcore-softcore}

\subsection{Hard-Core Constraint}\label{subsec:hardcore}

\begin{definition}[Hard-core anyons]\label{def:hardcore}
Anyons are \emph{hard-core} if at most one anyon occupies each site:
\begin{equation}
    x_1 < x_2 < \cdots < x_N
\end{equation}
\end{definition}

\begin{consequence}
Maximum particle number is $N_{\max} = n$ (number of sites).
\end{consequence}

\begin{remark}
Hard-core is natural for:
\begin{itemize}
    \item Impenetrable particles (infinite on-site repulsion)
    \item Lattice models where sites represent localised orbitals
\end{itemize}
\end{remark}

\subsection{Soft-Core: Multiple Occupancy}\label{subsec:softcore}

\begin{definition}[Soft-core anyons]\label{def:softcore}
Anyons are \emph{soft-core} if multiple anyons may occupy the same site:
\begin{equation}
    x_1 \leq x_2 \leq \cdots \leq x_N
\end{equation}
\end{definition}

\begin{definition}[On-site fusion space]\label{def:onsite-fusion}
When anyons $X_a, X_b$ occupy the same site, the local state space is:
\begin{equation}
    \bigoplus_c N_{ab}^c \cdot \Mor(X_a \otimes X_b, X_c)
\end{equation}
representing the possible fusion outcomes.
\end{definition}

\begin{remark}
Soft-core requires tracking on-site fusion structure.
\end{remark}

\subsection{Hilbert Space Modifications}\label{subsec:hilbert-mods}

\begin{center}
\begin{tabular}{llll}
\toprule
\textbf{Regime} & \textbf{Config space} & \textbf{Local structure} & $N_{\max}$ \\
\midrule
Hard-core & $x_i < x_j$ for $i < j$ & One anyon/site & $n$ \\
Soft-core & $x_i \leq x_j$ for $i < j$ & Fusion at each site & $\infty$ (needs cutoff) \\
\bottomrule
\end{tabular}
\end{center}

\begin{definition}[Hard-core Hilbert space]\label{def:hc-hilbert}
\begin{equation}
    \mathcal{H}^{\mathrm{HC}} = \bigoplus_{N=0}^{n} \mathcal{H}_N^{\mathrm{HC}}
\end{equation}
where $\mathcal{H}_N^{\mathrm{HC}}$ uses $\mathrm{Conf}_N^{\mathrm{HC}}$.
\end{definition}

\subsection{Physical Motivation}\label{subsec:physical-motivation}

\paragraph{Hard-core regime:}
\begin{itemize}
    \item Models impenetrable anyons
    \item Connection to Girardeau mapping (\S\ref{sec:girardeau})
    \item Simpler Hilbert space structure
\end{itemize}

\paragraph{Soft-core regime:}
\begin{itemize}
    \item Models ``bosonic'' anyons that can bunch
    \item Richer on-site physics
    \item Connection to Levin--Wen models when particles can annihilate
\end{itemize}

\begin{convention}
Unless stated otherwise, we work in the \textbf{hard-core} regime.
\end{convention}
