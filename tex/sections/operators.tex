% operators.tex
% LaTeX counterpart of docs/operators.md
% Section §4.3

\section{Operators as Morphisms}\label{sec:operators}

\textbf{Planning ref:} \S4.3\\
\textbf{Dependency note:} This section is logically prior to \S\ref{sec:hilbert-space}. We define operators as abstract categorical morphisms first, then show in \S\ref{sec:hilbert-space} how they act on the concrete Hilbert space.

\begin{assumption}\label{ass:operators}
\begin{enumerate}[label=(A4.3.\arabic*)]
    \item We work with finite tensor products of simple objects.
    \item The fusion category $\catC$ has finitely many simple objects.
\end{enumerate}
\end{assumption}

%=============================================================
\subsection{Abstract Operator Definition (Category-Level)}\label{subsec:abstract-operators}
%=============================================================

\begin{definition}[Morphism space operator]\label{def:morphism-operator}
An \emph{operator} is an element of a morphism space:
\begin{equation}
    \catO \in \Mor(A, B)
\end{equation}
where $A, B$ are tensor products of simple objects from $\catC$.
\end{definition}

\begin{remark}
This is a purely categorical definition, independent of any representation or Hilbert space. Morphisms are abstract; they follow categorical axioms (composition, associativity, identities).
\end{remark}

\begin{definition}[Operator as morphism sum]\label{def:operator-morphism-sum}
The space of all operators in this categorical sense is:
\begin{equation}
    \mathfrak{Op}_{\mathrm{cat}} = \bigoplus_{n_A, n_B \in \mathbb{Z}_{\ge 0}} \bigoplus_{\substack{(a_1, \ldots, a_{n_A}) \\ a_i \in \{1,\ldots,d-1\}}} \bigoplus_{\substack{(b_1, \ldots, b_{n_B}) \\ b_j \in \{1,\ldots,d-1\}}} \Mor(X_{a_1} \otimes \cdots \otimes X_{a_{n_A}}, X_{b_1} \otimes \cdots \otimes X_{b_{n_B}})
\end{equation}
where the nested direct sums run over:
\begin{itemize}
    \item $n_A, n_B \in \mathbb{Z}_{\ge 0}$ --- number of non-vacuum factors
    \item Labels $a_i, b_j \in \{1, \ldots, d-1\}$ --- indices of simple objects (excluding vacuum $X_0 = \one$)
\end{itemize}
By semisimplicity, $\Mor(A, B)$ is a finite-dimensional vector space for each choice of $A, B$.
\end{definition}

\begin{remark}
Objects with $n_A = 0$ are identified with $\one$ (tensor unit), so $X_\emptyset = \one$. This definition is independent of the Hilbert space $\catH$ and does not require it to be defined yet.
\end{remark}

%=============================================================
\subsection{Action on Hilbert Space (Concrete Representation)}\label{subsec:hilbert-action}
%=============================================================

Once the Hilbert space $\catH$ is defined (Definition~\ref{def:total-hilbert}), morphism operators are promoted to linear maps via a representation:

\begin{definition}[Representation of operators on $\catH$]\label{def:operator-representation}
A \emph{representation} of the morphism space $\Mor(A, B)$ on the Hilbert space $\catH$ is a linear embedding:
\begin{equation}
    \rho: \Mor(A, B) \to \mathrm{Lin}(\catH_A \to \catH_B)
\end{equation}
where $\catH_A$ denotes the sector of $\catH$ corresponding to object $A$, and $\mathrm{Lin}$ denotes linear maps.

The full operator algebra on $\catH$ is:
\begin{equation}
    \End(\catH) = \bigoplus_{A,B} \mathrm{Lin}(\catH_A \to \catH_B)
\end{equation}
\end{definition}

%=============================================================
\subsection{Particle-Number Conservation (at Category Level)}\label{subsec:particle-conservation}
%=============================================================

\begin{definition}[Particle-number of object]\label{def:particle-number-object}
For an object $A = X_{a_1} \otimes \cdots \otimes X_{a_n}$, define:
\begin{equation}
    \mathrm{N}(A) = n
\end{equation}
(the number of nontrivial factors, excluding vacuum $\one$).
\end{definition}

\begin{definition}[Particle-conserving morphism]\label{def:particle-conserving}
A morphism $\phi \in \Mor(A, B)$ is \emph{particle-conserving} if:
\begin{equation}
    \mathrm{N}(A) = \mathrm{N}(B)
\end{equation}
\end{definition}

\begin{definition}[Particle-changing morphism]\label{def:particle-changing}
A morphism $\phi \in \Mor(A, B)$ is \emph{particle-changing} if:
\begin{equation}
    \mathrm{N}(A) \neq \mathrm{N}(B)
\end{equation}
\end{definition}

\begin{example}
\begin{itemize}
    \item $\Mor(X_a \otimes X_b, X_c)$ is particle-annihilating: $\mathrm{N}(A) = 2$, $\mathrm{N}(B) = 1$.
    \item $\Mor(X_a, X_b \otimes X_c)$ is particle-creating: $\mathrm{N}(A) = 1$, $\mathrm{N}(B) = 2$.
\end{itemize}
\end{example}

%=============================================================
\subsection{Locality}\label{subsec:locality}
%=============================================================

\begin{definition}[Support of object]\label{def:support-object}
For an object $A = X_{a_1} \otimes \cdots \otimes X_{a_n}$, define:
\begin{equation}
    \mathrm{supp}(A) = \{j : a_j \neq 0\}
\end{equation}
(the set of factor positions that are nontrivial).
\end{definition}

\begin{definition}[$k$-local morphism]\label{def:k-local}
A morphism $\phi \in \Mor(A, B)$ is \emph{$k$-local} if the number of sites (factor positions) involved in the transition is at most $k$. Formally, this is satisfied if:
\begin{equation}
    |\mathrm{supp}(A) \cup \mathrm{supp}(B)| \leq k
\end{equation}
where $A$ and $B$ are viewed as objects on a local subset of sites.
\end{definition}

\begin{remark}[Well-definedness and embedding]
This definition is well-defined for \emph{primitive} morphisms acting on a small number of anyons. When such a morphism is embedded into a larger lattice $\Lambda$ with existing anyons at sites $j \notin \mathrm{supp}(A) \cup \mathrm{supp}(B)$, it is understood to act as the identity on those ``spectator'' anyons. Thus, a 2-local hopping morphism remains 2-local even when acting on a many-particle state, as its non-trivial action is restricted to $k=2$ sites.
\end{remark}

\begin{example}
\begin{itemize}
    \item $\Mor(X_a \otimes X_b, X_c \otimes X_d)$ is 2-local: only factors 0 and 1 are involved.
    \item $\Mor(X_a, X_a)$ is 1-local: only factor 0 is involved.
    \item $\Mor(X_a \otimes \one, \one \otimes X_a)$ is 2-local: acts on positions 0 and 1.
\end{itemize}
\end{example}

\begin{remark}
In the context of lattice systems, locality also specifies which lattice sites the morphism acts on. This is covered when embedding into the lattice in \S\ref{sec:hamiltonian-v0}.
\end{remark}
