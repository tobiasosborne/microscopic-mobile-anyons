% boundary_conditions.tex
% LaTeX counterpart of docs/boundary_conditions.md
% Section §6.5

\section{Boundary Conditions for Anyonic Chains}\label{sec:boundary-conditions}

\begin{assumption}\label{ass:boundary-conditions}
\begin{enumerate}[label=(A\arabic*)]
    \item Bulk fusion category $\catC$ (from \S\ref{sec:fusion-categories}).
    \item 1D chain with open boundary conditions (from \S\ref{sec:lattice}).
    \item Boundary conditions classified by $\catC$-module categories.
\end{enumerate}
\end{assumption}

\subsection{Overview}

For anyonic chains with open boundary conditions, the choice of \emph{boundary conditions} at each end significantly affects the Hilbert space structure and dynamics. Following the Kitaev--Kong framework, boundary conditions are classified by \emph{module categories} over the bulk fusion category $\catC$.

Different boundary conditions lead to different edge mode structures, affecting ground state degeneracy, edge excitations, and partition functions.

\subsection{Boundary Hilbert Space}

\begin{definition}[Boundary Hilbert space]\label{def:boundary-hilbert}
For a chain with:
\begin{itemize}
    \item Bulk category $\catC$
    \item Left boundary condition $\mathcal{M}_L$ (left $\catC$-module category)
    \item Right boundary condition $\mathcal{M}_R$ (right $\catC$-module category)
\end{itemize}
The boundary-modified Hilbert space involves the \emph{relative tensor product} $\mathcal{M}_L \boxtimes_\catC \mathcal{M}_R$.
\end{definition}

\subsection{Trivial Boundary Conditions}

\begin{definition}[Trivial/smooth boundary]\label{def:trivial-boundary}
The \emph{trivial boundary condition} corresponds to the regular module $\catC_\catC$, where $\catC$ acts on itself by tensor product.
\end{definition}

Anyons can freely approach the boundary without restriction. No edge modes beyond the bulk structure.

\subsection{Gapped Boundaries via Lagrangian Algebras}

\begin{definition}[Lagrangian algebra]\label{def:lagrangian-algebra}
A \emph{Lagrangian algebra} $A \in \catC$ is a commutative algebra object satisfying:
\begin{equation}
    \dim(A)^2 = \dim(\catC),
\end{equation}
where $\dim(\catC) = \sum_i d_i^2$ is the total quantum dimension.
\end{definition}

\begin{theorem}[Classification of gapped boundaries]\label{thm:gapped-boundaries}
Gapped boundary conditions for a topological phase with bulk $\catC$ are in bijection with Lagrangian algebras in $\catC$ (for modular $\catC$).
\end{theorem}

\begin{citationblock}
Kong--Wen, \emph{JHEP} (2014) \unverified
\end{citationblock}

A Lagrangian algebra specifies which bulk anyons can ``condense'' at the boundary.

\subsection{Examples}

\begin{example}[Fibonacci anyons]\label{ex:fib-boundary}
For the Fibonacci category $\catC = \mathrm{Fib}$ with simples $\{\one, \tau\}$:
\begin{center}
\begin{tabular}{@{}lll@{}}
\toprule
Module category & Simple objects & Physical meaning \\
\midrule
$\mathrm{Fib}_\mathrm{Fib}$ (regular) & $\{\one, \tau\}$ & Smooth boundary \\
$\mathrm{Vec}$ (condensed) & $\{\one\}$ & $\tau$ condensed at boundary \\
\bottomrule
\end{tabular}
\end{center}
\end{example}

\begin{example}[Ising anyons]\label{ex:ising-boundary}
For the Ising category $\catC = \mathrm{Ising}$ with simples $\{\one, \sigma, \psi\}$:
\begin{center}
\begin{tabular}{@{}lll@{}}
\toprule
Module category & Simple objects & Physical meaning \\
\midrule
$\mathrm{Ising}_\mathrm{Ising}$ & $\{\one, \sigma, \psi\}$ & Smooth boundary \\
$\mathrm{Vec}(\mathbb{Z}_2)$ & $\{\one, \psi\}$ & $\sigma$ condensed \\
\bottomrule
\end{tabular}
\end{center}
\end{example}

\subsection{Application to Mobile Anyons}

For mobile anyons on an open chain (our setting from \S\ref{sec:hilbert-space}):

\begin{enumerate}
    \item \textbf{Standard construction} (\S\ref{sec:hilbert-space}): Uses trivial boundary conditions implicitly.

    \item \textbf{With general boundaries:} The Hilbert space becomes:
    \begin{equation}
        \catH = \bigoplus_{M_L \in \mathcal{M}_L} \bigoplus_{M_R \in \mathcal{M}_R} \catH(M_L, M_R),
    \end{equation}
    where $\catH(M_L, M_R)$ is the space of bulk configurations interpolating between boundary states.

    \item \textbf{Boundary Hamiltonians:} Additional terms can describe:
    \begin{itemize}
        \item Boundary potentials (energy cost for edge modes)
        \item Boundary-bulk coupling (anyons interacting with edge)
        \item Boundary-changing operators (transitions between boundary conditions)
    \end{itemize}
\end{enumerate}

\subsection{Connection to Golden Chain}

\begin{example}[Golden chain boundaries]\label{ex:golden-chain-bc}
The golden chain (Fibonacci anyons at unit filling) has been studied with various boundary conditions:
\begin{itemize}
    \item \textbf{Free boundaries:} Regular module, leading to CFT edge modes.
    \item \textbf{Fixed boundaries:} Specific module object pinned at edge, breaking some symmetry.
\end{itemize}
\end{example}

\begin{citationblock}
Aasen--Fendley--Mong, \emph{J.\ Phys.\ A} (2020) \unverified
\end{citationblock}
