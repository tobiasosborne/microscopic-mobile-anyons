\section{Particle-Conserving Local Hamiltonians}
\label{sec:hamiltonian-v0}

\textbf{Planning ref:} \S5.1.1\\
\textbf{Status:} Draft

\begin{assumption}
A5.1.1. Fusion category $\mathcal{C}$ with unit $\mathbf{1}$ and simples $X_0=\mathbf{1}, X_1, \ldots, X_{d-1}$

A5.1.2. $n$ lattice sites labelled $0, \ldots, n-1$, OBC

A5.1.3. Hard-core regime: at most one anyon per site

A5.1.4. Hilbert space $\mathcal{H} = \bigoplus_{N=0}^{n} \mathcal{H}_N$ as in \S4.2
\end{assumption}

\subsection{Number-Conserving Hamiltonians (§5.1.1.1)}

\begin{definition}[Number operator]
The \textit{number operator} $\hat{N} : \mathcal{H} \to \mathcal{H}$ acts as
\begin{equation}
\hat{N} \ket{\psi} = N \ket{\psi} \quad \text{for } \ket{\psi} \in \mathcal{H}_N
\end{equation}
\end{definition}

\begin{definition}[Number-conserving Hamiltonian]
A Hamiltonian $H$ is \textit{number-conserving} (or \textit{particle-conserving}) if it commutes with the number operator:
\begin{equation}
[\hat{N}, H] = 0
\end{equation}
Equivalently, $H$ preserves each $N$-particle sector:
\begin{equation}
H : \mathcal{H}_N \to \mathcal{H}_N \quad \text{for all } N \in \{0, 1, \ldots, n\}
\end{equation}
\end{definition}

\begin{remark}
Number-conserving Hamiltonians do not create or annihilate anyons. They may move existing anyons between sites or introduce interactions between them.
\end{remark}

\subsection{Local Hamiltonians (§5.1.1.2)}

\begin{definition}[Local Hamiltonian]
A Hamiltonian $H$ is \textit{local} if it decomposes as
\begin{equation}
H = \sum_{j=0}^{n-2} h_j
\end{equation}
where each $h_j$ acts nontrivially only on a bounded neighbourhood of site $j$.
\end{definition}

\begin{definition}[Nearest-neighbour Hamiltonian]
A Hamiltonian $H$ is \textit{nearest-neighbour} if each local term $h_j$ acts nontrivially only on sites $j$ and $j+1$:
\begin{equation}
h_j : \mathcal{H} \to \mathcal{H}, \quad h_j = \mathbb{1}_{<j} \otimes \tilde{h}_j \otimes \mathbb{1}_{>j+1}
\end{equation}
where $\tilde{h}_j$ acts on the two-site space.
\end{definition}

\subsection{Morphism Decomposition (§5.1.1.3)}

For mobile anyons, local terms decompose into morphism spaces. A number-conserving, nearest-neighbour term $h_j$ has components:

\begin{definition}[Morphism components of local term]
A nearest-neighbour term $h_j$ acting on sites $j, j+1$ decomposes as:
\begin{equation}
h_j \in \bigoplus_{A,B} \Mor(A, B)
\end{equation}
where $A, B$ are objects of the form $X_a \otimes X_b$ with $a, b \in \{0, 1, \ldots, d-1\}$, and the sum is over pairs $(A,B)$ with equal numbers of nontrivial (non-vacuum) factors.
\end{definition}

\begin{proposition}[Number conservation criterion]
A local term $h_j$ is number-conserving if and only if for every nonzero component in $\Mor(X_a \otimes X_b, X_c \otimes X_d)$:
\begin{equation}
|\{a,b\} \cap \{1,\ldots,d-1\}| = |\{c,d\} \cap \{1,\ldots,d-1\}|
\end{equation}
where $|S|$ denotes the count of indices in $S$ that are nonzero (nontrivial).
\end{proposition}

\subsection{Classification of Two-Site Processes (§5.1.1.4)}

For hard-core anyons, the possible two-site configurations and number-conserving transitions are:

\begin{table}[h]
\centering
\begin{tabular}{|c|c|c|c|}
\hline
\textbf{Source $X_a \otimes X_b$} & \textbf{Target $X_c \otimes X_d$} & \textbf{Process} & \textbf{Particle count} \\
\hline
$\mathbf{1} \otimes \mathbf{1}$ & $\mathbf{1} \otimes \mathbf{1}$ & Vacuum identity & 0 \\
$X_a \otimes \mathbf{1}$ & $\mathbf{1} \otimes X_a$ & Hop right & 1 \\
$\mathbf{1} \otimes X_a$ & $X_a \otimes \mathbf{1}$ & Hop left & 1 \\
$X_a \otimes \mathbf{1}$ & $X_a \otimes \mathbf{1}$ & Stay left & 1 \\
$\mathbf{1} \otimes X_a$ & $\mathbf{1} \otimes X_a$ & Stay right & 1 \\
$X_a \otimes X_b$ & $X_a \otimes X_b$ & Two-anyon identity & 2 \\
$X_a \otimes X_b$ & $X_c \otimes X_d$ & Two-anyon scattering & 2 \\
\hline
\end{tabular}
\end{table}

\begin{remark}
Note that hopping does NOT involve braiding: the anyon moves to an empty site without passing through another anyon. Braiding processes are treated in \S5.1.3.
\end{remark}

\subsection{Hermiticity (§5.1.1.5)}

\begin{definition}[Hermitian local term]
A local term $h_j$ is \textit{Hermitian} if $h_j = h_j^\dagger$.

For morphism components, this means:
\begin{equation}
\langle B | h_j | A \rangle = \overline{\langle A | h_j | B \rangle}
\end{equation}
where the bar denotes complex conjugation.
\end{definition}

\begin{proposition}[Hermiticity in terms of morphisms]
A local term with component $\alpha \in \Mor(A, B)$ is Hermitian if and only if it also has component $\alpha^\dagger \in \Mor(B, A)$ with matching coefficient.
\end{proposition}

\subsection{Julia Implementation}

\begin{lstlisting}
# file: src/julia/MobileAnyons/hamiltonian_v0.jl
using LinearAlgebra

"""
    NumberConservingTerm

A nearest-neighbour term that conserves particle number.
Acts on sites (site, site+1).
"""
struct NumberConservingTerm
    site::Int                           # left site index (0-based)
    components::Dict{Tuple{Tuple{Int,Int}, Tuple{Int,Int}}, ComplexF64}
    # (source, target) => coefficient
    # source/target are (label_left, label_right), 0 = vacuum
end

"""
Check that all components conserve particle number.
"""
function is_number_conserving(term::NumberConservingTerm)
    for ((a, b), (c, d)) in keys(term.components)
        n_source = (a != 0) + (b != 0)
        n_target = (c != 0) + (d != 0)
        if n_source != n_target
            return false
        end
    end
    return true
end

"""
Check that the term is Hermitian.
"""
function is_hermitian(term::NumberConservingTerm)
    for (k, v) in term.components
        ((a,b), (c,d)) = k
        conj_key = ((c,d), (a,b))
        if !haskey(term.components, conj_key)
            return false
        end
        if term.components[conj_key] != conj(v)
            return false
        end
    end
    return true
end

"""
    LocalHamiltonian

A number-conserving nearest-neighbour Hamiltonian.
"""
struct LocalHamiltonian
    n_sites::Int
    terms::Vector{NumberConservingTerm}
end

"""
Build a uniform nearest-neighbour Hamiltonian from a single local term.
"""
function uniform_nn_hamiltonian(n_sites::Int, local_components::Dict)
    terms = [NumberConservingTerm(j, local_components) for j in 0:(n_sites-2)]
    return LocalHamiltonian(n_sites, terms)
end
\end{lstlisting}

\subsection{Summary}

\begin{table}[h]
\centering
\begin{tabular}{|c|c|c|}
\hline
\textbf{Concept} & \textbf{Symbol} & \textbf{Definition} \\
\hline
Number operator & $\hat{N}$ & Counts anyons: $\hat{N}|\psi\rangle = N|\psi\rangle$ for $|\psi\rangle \in \mathcal{H}_N$ \\
Number-conserving & $[\hat{N}, H] = 0$ & Preserves particle number \\
Nearest-neighbour & $h_j$ & Acts on sites $j, j+1$ only \\
Morphism component & $\Mor(A, B)$ & Transitions between local configurations \\
\hline
\end{tabular}
\end{table}

\subsection{Next Steps}

\begin{itemize}
\item \S5.1.1.2: Laplacian-type models (free hopping)
\item \S5.1.1.3: Hard-core blocking behaviour
\item \S5.1.2: Interactions without braiding
\item \S5.1.3: Free anyons with braiding
\end{itemize}
