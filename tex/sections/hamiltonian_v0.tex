\section{Particle-Conserving Local Hamiltonians}
\label{sec:hamiltonian-v0}

\textbf{Planning ref:} \S5.1.1\\
\textbf{Status:} Draft

\begin{assumption}
A5.1.1. Fusion category $\mathcal{C}$ with unit $\mathbf{1}$ and simples $X_0=\mathbf{1}, X_1, \ldots, X_{d-1}$

A5.1.2. $n$ lattice sites labelled $0, \ldots, n-1$, OBC

A5.1.3. Hard-core regime: at most one anyon per site

A5.1.4. Hilbert space $\mathcal{H} = \bigoplus_{N=0}^{n} \mathcal{H}_N$ as in \S4.2
\end{assumption}

\subsection{Number-Conserving Hamiltonians (§5.1.1.1)}

\begin{definition}[Number operator]
The \textit{number operator} $\hat{N} : \mathcal{H} \to \mathcal{H}$ acts as
\begin{equation}
\hat{N} \ket{\psi} = N \ket{\psi} \quad \text{for } \ket{\psi} \in \mathcal{H}_N
\end{equation}
\end{definition}

\begin{definition}[Number-conserving Hamiltonian]
A Hamiltonian $H$ is \textit{number-conserving} (or \textit{particle-conserving}) if it commutes with the number operator:
\begin{equation}
[\hat{N}, H] = 0
\end{equation}
Equivalently, $H$ preserves each $N$-particle sector:
\begin{equation}
H : \mathcal{H}_N \to \mathcal{H}_N \quad \text{for all } N \in \{0, 1, \ldots, n\}
\end{equation}
\end{definition}

\begin{remark}
Number-conserving Hamiltonians do not create or annihilate anyons. They may move existing anyons between sites or introduce interactions between them.
\end{remark}

\subsection{Local Hamiltonians (§5.1.1.2)}

\begin{definition}[Local Hamiltonian]
A Hamiltonian $H$ is \textit{local} if it decomposes as
\begin{equation}
H = \sum_{j=0}^{n-2} h_j
\end{equation}
where each $h_j$ acts nontrivially only on a bounded neighbourhood of site $j$.
\end{definition}

\begin{definition}[Nearest-neighbour Hamiltonian]
A Hamiltonian $H$ is \textit{nearest-neighbour} if each local term $h_j$ acts nontrivially only on sites $j$ and $j+1$:
\begin{equation}
h_j : \mathcal{H} \to \mathcal{H}, \quad h_j = \mathbb{1}_{<j} \otimes \tilde{h}_j \otimes \mathbb{1}_{>j+1}
\end{equation}
where $\tilde{h}_j$ acts on the two-site space.
\end{definition}

\subsection{Morphism Decomposition (§5.1.1.3)}

For mobile anyons, local terms decompose into morphism spaces. A number-conserving, nearest-neighbour term $h_j$ has components:

\begin{definition}[Morphism components of local term]
A nearest-neighbour term $h_j$ acting on sites $j, j+1$ decomposes as:
\begin{equation}
h_j \in \bigoplus_{A,B} \Mor(A, B)
\end{equation}
where $A, B$ are objects of the form $X_a \otimes X_b$ with $a, b \in \{0, 1, \ldots, d-1\}$, and the sum is over pairs $(A,B)$ with equal numbers of nontrivial (non-vacuum) factors.
\end{definition}

\begin{proposition}[Number conservation criterion]
A local term $h_j$ is number-conserving if and only if for every nonzero component in $\Mor(X_a \otimes X_b, X_c \otimes X_d)$:
\begin{equation}
|\{a,b\} \cap \{1,\ldots,d-1\}| = |\{c,d\} \cap \{1,\ldots,d-1\}|
\end{equation}
where $|S|$ denotes the count of indices in $S$ that are nonzero (nontrivial).
\end{proposition}

\subsection{Classification of Two-Site Processes (§5.1.1.4)}

For hard-core anyons, the possible two-site configurations and number-conserving transitions are:

\begin{table}[h]
\centering
\begin{tabular}{|c|c|c|c|}
\hline
\textbf{Source $X_a \otimes X_b$} & \textbf{Target $X_c \otimes X_d$} & \textbf{Process} & \textbf{Particle count} \\
\hline
$\mathbf{1} \otimes \mathbf{1}$ & $\mathbf{1} \otimes \mathbf{1}$ & Vacuum identity & 0 \\
$X_a \otimes \mathbf{1}$ & $\mathbf{1} \otimes X_a$ & Hop right & 1 \\
$\mathbf{1} \otimes X_a$ & $X_a \otimes \mathbf{1}$ & Hop left & 1 \\
$X_a \otimes \mathbf{1}$ & $X_a \otimes \mathbf{1}$ & Stay left & 1 \\
$\mathbf{1} \otimes X_a$ & $\mathbf{1} \otimes X_a$ & Stay right & 1 \\
$X_a \otimes X_b$ & $X_a \otimes X_b$ & Two-anyon identity & 2 \\
$X_a \otimes X_b$ & $X_c \otimes X_d$ & Two-anyon scattering & 2 \\
\hline
\end{tabular}
\end{table}

\begin{remark}
Note that hopping does NOT involve braiding: the anyon moves to an empty site without passing through another anyon. Braiding processes are treated in \S5.1.3.
\end{remark}

\subsection{Hermiticity (§5.1.1.5)}

\begin{definition}[Hermitian local term]
A local term $h_j$ is \textit{Hermitian} if $h_j = h_j^\dagger$.

For morphism components, this means:
\begin{equation}
\langle B | h_j | A \rangle = \overline{\langle A | h_j | B \rangle}
\end{equation}
where the bar denotes complex conjugation.
\end{definition}

\begin{proposition}[Hermiticity in terms of morphisms]
A local term with component $\alpha \in \Mor(A, B)$ is Hermitian if and only if it also has component $\alpha^\dagger \in \Mor(B, A)$ with matching coefficient.
\end{proposition}

\subsection{Laplacian-Type Models (§5.1.1.2)}
\label{subsec:laplacian}

The simplest number-conserving Hamiltonians are \emph{tight-binding} or \emph{Laplacian-type} models, where anyons hop freely between adjacent sites.

\subsubsection{Single-Particle Sector}

\begin{definition}[Anyonic discrete Laplacian]\label{def:anyonic-laplacian}
The \emph{anyonic discrete Laplacian} on the single-particle sector $\mathcal{H}_1$ is:
\begin{equation}\label{eq:laplacian-single}
    H_{\mathrm{Lap}} = -t \sum_{j=0}^{n-2} \sum_{a=1}^{d-1} \left( \ket{j+1; a}\bra{j; a} + \ket{j; a}\bra{j+1; a} \right)
\end{equation}
where $t > 0$ is the hopping amplitude and $\ket{j; a}$ denotes a single anyon of type $X_a$ at site $j$.
\end{definition}

\begin{remark}
In the single-particle sector, there is no distinction between anyonic and bosonic behaviour: a lone anyon hops without braiding or fusion interactions. The Laplacian preserves the anyon type $a$.
\end{remark}

\begin{proposition}[Spectrum of single-particle Laplacian]\label{prop:laplacian-spectrum}
For OBC on $n$ sites, the single-particle Laplacian has spectrum:
\begin{equation}
    E_k = -2t \cos\left( \frac{\pi k}{n+1} \right), \quad k = 1, \ldots, n
\end{equation}
with eigenstates:
\begin{equation}
    \ket{\psi_k; a} = \sqrt{\frac{2}{n+1}} \sum_{j=0}^{n-1} \sin\left( \frac{\pi k (j+1)}{n+1} \right) \ket{j; a}
\end{equation}
Each energy level has $(d-1)$-fold degeneracy corresponding to anyon types.
\end{proposition}

\subsubsection{Multi-Particle Sector with Hard-Core Constraint}

For $N > 1$ anyons in the hard-core regime, hopping is only permitted to empty adjacent sites.

\begin{definition}[Hard-core hopping term]\label{def:hc-hopping}
The nearest-neighbour hopping term at bond $(j, j+1)$ for hard-core anyons is:
\begin{equation}\label{eq:hc-hopping}
    h_j^{\mathrm{hop}} = -t \sum_{a=1}^{d-1} \left( P_j^{(0)} \otimes P_{j+1}^{(a)} \to P_j^{(a)} \otimes P_{j+1}^{(0)} \right) + \mathrm{h.c.}
\end{equation}
where $P_m^{(a)}$ projects onto anyon type $X_a$ at site $m$, and $X_0 = \mathbf{1}$ is the vacuum.
\end{definition}

\begin{remark}
In morphism language, the hopping term has components in $\Mor(\mathbf{1} \otimes X_a, X_a \otimes \mathbf{1})$ and its adjoint. For the vacuum object $\mathbf{1}$, these morphism spaces are one-dimensional (canonical isomorphisms from the unit constraints), so the hopping amplitude is unambiguous.
\end{remark}

\begin{definition}[Hard-core tight-binding Hamiltonian]\label{def:hc-tight-binding}
The \emph{hard-core tight-binding Hamiltonian} is:
\begin{equation}\label{eq:hc-tight-binding}
    H_{\mathrm{TB}} = \sum_{j=0}^{n-2} h_j^{\mathrm{hop}}
\end{equation}
This Hamiltonian conserves particle number and respects the hard-core constraint.
\end{definition}

\subsubsection{Diagrammatic Representation}

The hopping term can be represented diagrammatically as:

\begin{center}
\begin{tikzpicture}[scale=0.8]
    % Left diagram: before hop
    \node at (-2, 1.5) {Before:};
    \draw[thick, dashed] (-0.5, 0) -- (-0.5, 1.5);
    \draw[thick, AnyonBlue] (0.5, 0) -- (0.5, 1.5);
    \node at (-0.5, -0.3) {$\mathbf{1}$};
    \node at (0.5, -0.3) {$X_a$};

    % Arrow
    \node at (2, 0.75) {$\xrightarrow{h_j^{\mathrm{hop}}}$};

    % Right diagram: after hop
    \node at (4.5, 1.5) {After:};
    \draw[thick, AnyonBlue] (3.5, 0) -- (3.5, 1.5);
    \draw[thick, dashed] (4.5, 0) -- (4.5, 1.5);
    \node at (3.5, -0.3) {$X_a$};
    \node at (4.5, -0.3) {$\mathbf{1}$};
\end{tikzpicture}
\end{center}

The dashed line represents the vacuum $\mathbf{1}$, and the solid line represents an anyon of type $X_a$.

\subsubsection{Connection to Standard Models}

\begin{proposition}[Bosonic limit]\label{prop:bosonic-limit}
For the category $\mathsf{Vec}$ (trivial fusion category with one simple object), $H_{\mathrm{TB}}$ reduces to the standard hard-core boson tight-binding model:
\begin{equation}
    H_{\mathrm{TB}} = -t \sum_{j} \left( b_j^\dagger b_{j+1} + b_{j+1}^\dagger b_j \right)(1 - n_j n_{j+1})
\end{equation}
where the $(1 - n_j n_{j+1})$ factor enforces hard-core exclusion.
\end{proposition}

\begin{proposition}[Fermionic limit]\label{prop:fermionic-limit}
For the category $\mathsf{sVec}$ (super-vector spaces), with appropriate sign conventions, $H_{\mathrm{TB}}$ reduces to the spinless fermion tight-binding model:
\begin{equation}
    H_{\mathrm{TB}} = -t \sum_{j} \left( c_j^\dagger c_{j+1} + c_{j+1}^\dagger c_j \right)
\end{equation}
The hard-core constraint is automatic due to Pauli exclusion.
\end{proposition}

\begin{remark}
These limiting cases demonstrate that the anyonic tight-binding model correctly interpolates between bosonic and fermionic physics in appropriate limits.
\end{remark}

\subsection{Julia Implementation}

\begin{lstlisting}
# file: src/julia/MobileAnyons/hamiltonian_v0.jl
using LinearAlgebra

"""
    NumberConservingTerm

A nearest-neighbour term that conserves particle number.
Acts on sites (site, site+1).
"""
struct NumberConservingTerm
    site::Int                           # left site index (0-based)
    components::Dict{Tuple{Tuple{Int,Int}, Tuple{Int,Int}}, ComplexF64}
    # (source, target) => coefficient
    # source/target are (label_left, label_right), 0 = vacuum
end

"""
Check that all components conserve particle number.
"""
function is_number_conserving(term::NumberConservingTerm)
    for ((a, b), (c, d)) in keys(term.components)
        n_source = (a != 0) + (b != 0)
        n_target = (c != 0) + (d != 0)
        if n_source != n_target
            return false
        end
    end
    return true
end

"""
Check that the term is Hermitian.
"""
function is_hermitian(term::NumberConservingTerm)
    for (k, v) in term.components
        ((a,b), (c,d)) = k
        conj_key = ((c,d), (a,b))
        if !haskey(term.components, conj_key)
            return false
        end
        if term.components[conj_key] != conj(v)
            return false
        end
    end
    return true
end

"""
    LocalHamiltonian

A number-conserving nearest-neighbour Hamiltonian.
"""
struct LocalHamiltonian
    n_sites::Int
    terms::Vector{NumberConservingTerm}
end

"""
Build a uniform nearest-neighbour Hamiltonian from a single local term.
"""
function uniform_nn_hamiltonian(n_sites::Int, local_components::Dict)
    terms = [NumberConservingTerm(j, local_components) for j in 0:(n_sites-2)]
    return LocalHamiltonian(n_sites, terms)
end

"""
    laplacian_hopping_components(d::Int, t::Real=1.0)

Build hopping components for the hard-core tight-binding Hamiltonian.
d = number of simple objects (including vacuum at index 0).
Returns Dict for use with uniform_nn_hamiltonian.
"""
function laplacian_hopping_components(d::Int, t::Real=1.0)
    components = Dict{Tuple{Tuple{Int,Int}, Tuple{Int,Int}}, ComplexF64}()
    for a in 1:(d-1)  # non-vacuum anyon types
        # Hop right: (0, a) -> (a, 0)
        components[((0, a), (a, 0))] = -t
        # Hop left: (a, 0) -> (0, a)  [Hermitian conjugate]
        components[((a, 0), (0, a))] = -t
    end
    return components
end

"""
    tight_binding_hamiltonian(n_sites::Int, d::Int, t::Real=1.0)

Build the hard-core tight-binding Hamiltonian for d simple objects.
"""
function tight_binding_hamiltonian(n_sites::Int, d::Int, t::Real=1.0)
    components = laplacian_hopping_components(d, t)
    return uniform_nn_hamiltonian(n_sites, components)
end
\end{lstlisting}

\subsection{Summary}

\begin{table}[h]
\centering
\begin{tabular}{|c|c|c|}
\hline
\textbf{Concept} & \textbf{Symbol} & \textbf{Definition} \\
\hline
Number operator & $\hat{N}$ & Counts anyons: $\hat{N}|\psi\rangle = N|\psi\rangle$ for $|\psi\rangle \in \mathcal{H}_N$ \\
Number-conserving & $[\hat{N}, H] = 0$ & Preserves particle number \\
Nearest-neighbour & $h_j$ & Acts on sites $j, j+1$ only \\
Morphism component & $\Mor(A, B)$ & Transitions between local configurations \\
Anyonic Laplacian & $H_{\mathrm{Lap}}$ & Free hopping on single-particle sector \\
Tight-binding & $H_{\mathrm{TB}}$ & Hard-core hopping on multi-particle sector \\
\hline
\end{tabular}
\end{table}

\subsection{Next Steps}

\begin{itemize}
\item \S5.1.1.3: Hard-core blocking behaviour
\item \S5.1.2: Interactions without braiding
\item \S5.1.3: Free anyons with braiding
\end{itemize}
