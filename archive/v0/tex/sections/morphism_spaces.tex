% morphism_spaces.tex
% LaTeX counterpart of docs/morphism_spaces.md
% Section §3.1.3

\section{Morphism Spaces and Multiplicities}\label{sec:morphism-spaces}

\begin{assumption}\label{ass:morphism-spaces}
\begin{enumerate}[label=(A3.1.3.\arabic*)]
    \item Fusion category $(\mathcal{C}, \otimes, \mathbf{1})$ over an algebraically closed field $k$ (Definition~\ref{def:fusion-category}).
    \item $\mathcal{C}$ is semisimple and $k$-linear, so all morphism spaces are finite-dimensional $k$-vector spaces.
\end{enumerate}
\end{assumption}

\begin{definition}[Morphism space]\label{def:morphism-space}
For any objects $A, B \in \mathcal{C}$,
\begin{equation}
    \Mor(A, B) := \Hom_{\mathcal{C}}(A, B)
\end{equation}
is a finite-dimensional $k$-vector space. If $A, B$ are simple, Schur's lemma implies $\dim \Mor(A, B) = \delta_{A, B}$.
\end{definition}

\begin{citationblock}
Etingof--Gelaki--Nikshych--Ostrik, \emph{Tensor Categories}, AMS (2015), \S4.2 \cite{EGNO2015} \unverified
\end{citationblock}

\begin{definition}[Fusion multiplicity space]\label{def:fusion-multiplicity-space}
For simple objects $X_a, X_b, X_c \in \mathrm{Irr}(\mathcal{C})$, the space
\begin{equation}
    \Mor(X_a \otimes X_b, X_c)
\end{equation}
has dimension $N_{ab}^c = \dim \Mor(X_a \otimes X_b, X_c) \in \mathbb{Z}_{\ge 0}$. A \emph{multiplicity basis} is any choice of morphisms
\begin{equation}
    f_{ab \to c}^{(\mu)} : X_a \otimes X_b \to X_c, \quad \mu = 1, \ldots, N_{ab}^c.
\end{equation}
No canonical choice exists; computations must remain basis-independent.
\end{definition}

\begin{claim}[Multiplicity-free simplification]\label{claim:mult-free}
In the multiplicity-free case ($N_{ab}^c \in \{0, 1\}$), each space $\Mor(X_a \otimes X_b, X_c)$ is either $\{0\}$ or a one-dimensional $k$-line. Basis dependence disappears, and $f_{ab \to c}^{(1)}$ can be chosen uniquely up to phase.
\end{claim}

\begin{remark}
Duals: $\Mor(\mathbf{1}, X_a \otimes X_b)$ is canonically dual to $\Mor(X_a^* \otimes X_b^*, \mathbf{1})$ via rigidity. Normalisation choices for evaluation/coevaluation maps must be consistent.
\end{remark}

\begin{remark}
Basis independence is essential for categorical definitions. Fusion-tree bases are admissible for computations (e.g., numerical evaluation of $F$-symbols) but must be removed from statements of definitions and theorems.
\end{remark}
